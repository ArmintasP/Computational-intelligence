\documentclass{VUMIFPSbakalaurinis}
\usepackage{float}
\usepackage{hyperref}
\usepackage{algorithmicx}
\usepackage{algorithm}
\usepackage{algpseudocode}
\usepackage{amsfonts}
\usepackage{amsmath}
\usepackage{bm}
\usepackage{caption}
\usepackage{color}
\usepackage{graphicx}
\usepackage{listings}
\usepackage{subcaption}
\usepackage{wrapfig}
\usepackage{biblatex}
\usepackage{microtype}
\usepackage{mathtools}
\usepackage{multirow}
\usepackage{graphics}


% Titulinio aprašas
\university{Vilniaus universitetss}
\faculty{Matematikos ir informatikos fakultetas}
\institute{Informatikos institutas}  % Užkomentavus šią eilutę - institutas neįtraukiamas į titulinį
\department{Programų sistemų bakalauro studijų programa}
\papertype{Laboratorinio darbo ataskaita}
\title{Saviorganizuojantys neuroniniai tinklai}
\titleineng{Self organizing neural networks}
\author{Armintas Pakenis}
% \secondauthor{Vardonis Pavardonis}   % Pridėti antrą autorių
\supervisor{prof. dr. Olga Kurasova}
\reviewer{prof. dr. Olga Kurasova}
% \addsignatureplaces{} % prideda parašų vietas tituliniame puslapyje
\date{Vilnius – \the\year}

\bibliography{bibliografija}

\begin{document}
\maketitle

%% Padėkų skyrius
% \sectionnonumnocontent{}
% \vspace{7cm}
% \begin{center}
%     Padėkos asmenims ir/ar organizacijoms
% \end{center}

\tableofcontents

\section{Užduoties tikslas}
Užduoties tikslas — suprogramuoti saviorganizuojančio neuroninio tinklo (žemėlapio, SOM)
mokymo algoritmą, apmokyti jį naudojant pasirinktus duomenis.
Naudotos programos kodą 
galima rasti GitHub repositorijoje: 
\href{https://github.com/ArmintasP/Computational-intelligence/tree/main/Lab5}{\color{cyan}{https://github.com/ArmintasP/Computational-intelligence/tree/main/Lab5}}.

\subsection{Duomenys}
Naudotas irisų duomenų rinkinys
\href{https://archive.ics.uci.edu/ml/datasets/iris}{\color{cyan}{https://archive.ics.uci.edu/ml/datasets/iris}}.

Irisų duomenų rinkinyje buvo 150 pavyzdžių,
iš kurių po 50 įrašų kiekvienos klasės (iš viso 3 klasės).
Požymių skaičius — 4.

\subsection{Įranga}
Modelio treniravimas vyko ant asmeninio kompiuterio su
AMD Ryzen 5 5600G APU (CPU) ir 32 GB RAM.

Taip pat ilgesniems skaičiavimas buvo naudoti
MIF superkompiuterio resursai.

\section{Metodai}

Saviorganizuojančio neuroninio žemėlapio
mokymas buvo atliekamas dviejais skirtingais mokymo būdais, 
tačiau naudojant tas pačias pagrindines skaičiavimo funkcijas.

\subsection{Naudotos funkcijos}
Atstumams apskaičiuoti buvo naudojama Euklido atstumų funkcija:

$$
d\left( x,y\right)   = \sqrt {\sum _{i=1}^{n}  \left( x_{i}-y_{i}\right)^2 } 
,$$
kur n — požymių skaičius, ${x}$ — duomenų 
rinkinio pavyzdžio vektorius, o ${y}$ - žemėlapio
neurono svorių vektorius.

Kaimynų atstumui įvertinti naudota Gauso kaimynų atsutumų funkcija:
$$
h_{ij}^{c}{t} = \alpha(t, T) \cdot \exp(\frac{-d(R_{c} - R_{ij}^2)}{2 \cdot (n_{ij}^c(t))^2})
,$$
kur $c$ — nugalėtojo neurono indeksas,
$t$ — iteracijos (epochos) numeris,
$T$ — iteracijų (epochų) skaičius,
$n_{ij}^c(t)$  — kaiminystės eilės numeris nugalėtojo atžvilgiu,
$R_c, R_ij$ — nugalėtojo neurono ir lyginimajo neurono indeksų vektoriai,
čia $\alpha(t, T) = 1 - \frac{t}{T}$.

\subsection{Mokymas}

Mokymas buvo atliktas 2 būdais. Pirmu būdu atliekant mokymą
duomenų rinkinys nebuvo permaišomas. Per vieną epochą
kiekvienam pavyzdžiui $X_i$ buvo randamas 
neuronas nugalėtojas ir atitinkamai pakeičiamos
žemėlapio neuronų reikšmės.

Antruoju būdu mokymas buvo atliekamas iteracijomis,
kur per kiekvieną iteraciją yra imamas atsitiktinis
vektorius $X_i$, kuriam randamas neuronas nugalėtojas
ir atitinkamai pakeičiamos žemėlapio neuronų reikšmės.


\section{Rezultatai}
1 mokymo būdu buvo atlikti 4 stebėjimai su kintančiu epochų
skaičiumi (žr. \ref{tab:q-1} lentelę). Matosi tendencija,
jog didėjant epochų skaičiui, mažėja kvantavimo paklaida.
Tiesa, mažėjimas pastebimas mažesnis arba nebepastebimas,
kai epochų skaičius pasieka 1000. Taip pat didėjant žemėlapio
dydžiui, mažėja ir kvantavimo paklaida, tačiau jei lygintumėme
$15 \times 15$ žemėlapį su $10 \times 10$ dydžio žemėlapį,
tai kvantavimo paklaidos pokytis nebuvo toks ryškus kaip
tarp $10 \times 10$ ir $5 \times 5$ dydžio SOM.

Antru mokymo būdu buvo atlikti 5 stebėjimai su kintančiu
iteracijų skaičiumi (žr. \ref{tab:q-2} lentelę). Galioja
tos pačios tendencijos kaip minėtos anksčiau aptariant 
\ref{tab:q-1} lentelę. Tiesa, labai išaugus iteracijų skaičiui,
(šiuo atveju pasiekus 1500000 iteracijų), kvantavimo paklaida
gavosi prastesnė nei su ankstesnėmis iteracijomis. Kitąvertus,
antru mokymu būdu pavyko pasiekti neženkliai geresnes arba vienodas
kvantavimo paklaidas kiekvienam žemėlapiui.

\ref{tab:5x5-1} ir \ref{tab:10x10-1} lentelėse pateikiamas
žemėlapis lentelės formatu, kur skaičiai žymi žemėlapio eilutės
ar stulpelio numerį. Žemėlapiui sukuti buvo naudojami tie patys
mokymo duomenys, kurie sudaryti iš trijų klasių.

Abejuose žemėlapiuose matyti, kad duomenys klasterizavosi
, o geriausiai tai nutiko su "Iris-setosa" ir "Iris-virginica"
klasių duomenimis. Šiuo atveju, saviorganizuojančio žemėlapio
dydis reikšmės neturėjo. Tačiau dera atkreipti dėmesį, kad
duomenų rinkinys ganėtinai paprastas, neturi daug atributų,
klasių ne daug, duomenų kiekis nedidelis. Visa tai leidžia
saviorganizuojantį neuorininį tinklą mokyti daug epochų,
kad mažai besiskiriantys dydžiu žemėlapiai stipriai varijuotų.

\begin{table}[]
  \centering
  \caption{1 mokymo būdu mokyto SOM kvantavimo paklaidos}{
    \begin{tabular}{|l|lll|l}
      \cline{1-4}
      \multirow{3}{*}{\textbf{Epochos}} & \multicolumn{3}{l|}{\textbf{Žemėlapio dydis}}                                            &  \\ \cline{2-4}
                                           & \multicolumn{1}{l|}{\textbf{5x5}} & \multicolumn{1}{l|}{\textbf{10x10}} & \textbf{15x15} &  \\ \cline{2-4}
            & \multicolumn{3}{l|}{Kvantavimo paklaida}                        &  \\ \cline{1-4}
      10    & \multicolumn{1}{l|}{2,23} & \multicolumn{1}{l|}{2,17} & 2,16    &  \\ \cline{1-4}
      50    & \multicolumn{1}{l|}{1,63} & \multicolumn{1}{l|}{1,52} & 1,51    &  \\ \cline{1-4}
      1000  & \multicolumn{1}{l|}{1,49} & \multicolumn{1}{l|}{1,39} & 1,35    &  \\ \cline{1-4}
      10000 & \multicolumn{1}{l|}{1,49} & \multicolumn{1}{l|}{1,38} & 1,34 &  \\ \cline{1-4}
      \end{tabular}}
  \label{tab:q-1}
\end{table}

\begin{table}[]
  \centering
  \caption{2 mokymo būdu mokyto SOM kvantavimo paklaidos}{
    \begin{tabular}{|l|lll|l}
      \cline{1-4}
      \multirow{3}{*}{\textbf{Iteracijos}} & \multicolumn{3}{l|}{\textbf{Žemėlapio dydis}}                                            &  \\ \cline{2-4}
                                           & \multicolumn{1}{l|}{\textbf{5x5}} & \multicolumn{1}{l|}{\textbf{10x10}} & \textbf{15x15} &  \\ \cline{2-4}
             & \multicolumn{3}{l|}{Kvantavimo paklaida}                     &  \\ \cline{1-4}
      50     & \multicolumn{1}{l|}{1,52} & \multicolumn{1}{l|}{1,36} & 1,43 &  \\ \cline{1-4}
      100    & \multicolumn{1}{l|}{1,5}  & \multicolumn{1}{l|}{1,44} & 1,40 &  \\ \cline{1-4}
      1500   & \multicolumn{1}{l|}{1,45} & \multicolumn{1}{l|}{1,39} & 1,39 &  \\ \cline{1-4}
      7500   & \multicolumn{1}{l|}{1,49} & \multicolumn{1}{l|}{1,38} & 1,32 &  \\ \cline{1-4}
      150000 & \multicolumn{1}{l|}{1,49} & \multicolumn{1}{l|}{1,39} & 1,36 &  \\ \cline{1-4}
      \end{tabular}}
  \label{tab:q-2}
\end{table}



\begin{table}[]
  \centering
  \caption{1 mokymo būdu mokyto 5x5 dydžio SOM žemėlapis
   po 1000 epochų}{
    \resizebox{\columnwidth}{!}{%  
    \begin{tabular}{|l|l|l|l|l|l|}
      \hline
      \textbf{}  & \textbf{1}                   & \textbf{2}      & \textbf{3}      & \textbf{4}                      & \textbf{5}                      \\ \hline
      \textbf{1} &                              &                 &                 &                                 & Iris-versicolor, Iris-virginica \\ \hline
      \textbf{2} &                              &                 &                 & Iris-versicolor, Iris-virginica & Iris-versicolor                 \\ \hline
      \textbf{3} &                              & Iris-versicolor & Iris-versicolor & Iris-versicolor                 &                                 \\ \hline
      \textbf{4} & Iris-versicolor              & Iris-versicolor &                 &                                 &                                 \\ \hline
      \textbf{5} & Iris-setosa, Iris-versicolor &                 &                 &                                 &                                 \\ \hline
      \end{tabular}}%
  }
  \label{tab:5x5-1}
\end{table}

\begin{table}[]
  \centering
  \caption{1 mokymo būdu mokyto 10x10 dydžio SOM žemėlapis po 1000 epochų}{
    \resizebox{\columnwidth}{!}{%  
    \begin{tabular}{|l|l|l|l|l|l|l|l|l|l|l|}
      \hline
      \textbf{} & \textbf{1} & \textbf{2} & \textbf{3} & \textbf{4} & \textbf{5} & \textbf{6} & \textbf{7} & \textbf{8} & \textbf{9} & \textbf{10} \\ \hline
      \textbf{1} & Iris-setosa, Iris-versicolor & Iris-versicolor &  &  &  &  &  &  &  &  \\ \hline
      \textbf{2} &  & Iris-versicolor &  &  &  &  &  &  &  &  \\ \hline
      \textbf{3} &  & Iris-versicolor & Iris-versicolor &  &  &  &  &  &  &  \\ \hline
      \textbf{4} &  &  &  & Iris-versicolor &  &  &  &  &  &  \\ \hline
      \textbf{5} &  &  &  &  & Iris-versicolor &  &  &  &  &  \\ \hline
      \textbf{6} &  &  &  &  &  & Iris-versicolor &  &  &  &  \\ \hline
      \textbf{7} &  &  &  &  &  &  & Iris-versicolor &  &  &  \\ \hline
      \textbf{8} &  &  &  &  &  &  &  & Iris-versicolor, Iris-virginica &  &  \\ \hline
      \textbf{9} &  &  &  &  &  &  &  & Iris-versicolor & Iris-versicolor & Iris-versicolor \\ \hline
      \textbf{10} &  &  &  &  &  &  &  &  &  & Iris-versicolor, Iris-virginica \\ \hline
      \end{tabular}}%
  }
  \label{tab:10x10-1}
\end{table}


\section{Išvados}

Kvantavimo paklaida priklauso nuo epochų (iteracijų) skaičiaus,
pasirinkto SOM mokymo būdo (algoritmo). Paklaidos mažėjimas 
taip pat priklausomas ir nuo žemėlapio didėjimo, tačiau stipriai 
didinant žemėlapio matmenis paklaidos mažėjimas ima lėtėti. 
Su irisų duomenų rinkiniu duomenys aiškiai klasterizavosi,
tačiau "iris-versicolor" buvo labiausiai išsisklaidęs. Skirtingo
dydžio apmokyti žemėlapiai savo rezultatais neišskysrė stipriai,
tiek $10 \times 10$, tiek $5 \times 5$ dydžio SOM rezultatų 
interpretacijos yra artimos.

\end{document}
